
%% bare_jrnl_compsoc.tex
%% V1.4b
%% 2015/08/26
%% by Michael Shell
%% See:
%% http://www.michaelshell.org/
%% for current contact information.
%%
%% This is a skeleton file demonstrating the use of IEEEtran.cls
%% (requires IEEEtran.cls version 1.8b or later) with an IEEE
%% Computer Society journal paper.
%%
%% Support sites:
%% http://www.michaelshell.org/tex/ieeetran/
%% http://www.ctan.org/pkg/ieeetran
%% and
%% http://www.ieee.org/

%%*************************************************************************
%% Legal Notice:
%% This code is offered as-is without any warranty either expressed or
%% implied; without even the implied warranty of MERCHANTABILITY or
%% FITNESS FOR A PARTICULAR PURPOSE! 
%% User assumes all risk.
%% In no event shall the IEEE or any contributor to this code be liable for
%% any damages or losses, including, but not limited to, incidental,
%% consequential, or any other damages, resulting from the use or misuse
%% of any information contained here.
%%
%% All comments are the opinions of their respective authors and are not
%% necessarily endorsed by the IEEE.
%%
%% This work is distributed under the LaTeX Project Public License (LPPL)
%% ( http://www.latex-project.org/ ) version 1.3, and may be freely used,
%% distributed and modified. A copy of the LPPL, version 1.3, is included
%% in the base LaTeX documentation of all distributions of LaTeX released
%% 2003/12/01 or later.
%% Retain all contribution notices and credits.
%% ** Modified files should be clearly indicated as such, including  **
%% ** renaming them and changing author support contact information. **
%%*************************************************************************


% *** Authors should verify (and, if needed, correct) their LaTeX system  ***
% *** with the testflow diagnostic prior to trusting their LaTeX platform ***
% *** with production work. The IEEE's font choices and paper sizes can   ***
% *** trigger bugs that do not appear when using other class files.       ***                          ***
% The testflow support page is at:
% http://www.michaelshell.org/tex/testflow/


\documentclass[10pt,journal,compsoc]{IEEEtran}
%
% If IEEEtran.cls has not been installed into the LaTeX system files,
% manually specify the path to it like:
% \documentclass[10pt,journal,compsoc]{../sty/IEEEtran}





% Some very useful LaTeX packages include:
% (uncomment the ones you want to load)


% *** MISC UTILITY PACKAGES ***
%
%\usepackage{ifpdf}
% Heiko Oberdiek's ifpdf.sty is very useful if you need conditional
% compilation based on whether the output is pdf or dvi.
% usage:
% \ifpdf
%   % pdf code
% \else
%   % dvi code
% \fi
% The latest version of ifpdf.sty can be obtained from:
% http://www.ctan.org/pkg/ifpdf
% Also, note that IEEEtran.cls V1.7 and later provides a builtin
% \ifCLASSINFOpdf conditional that works the same way.
% When switching from latex to pdflatex and vice-versa, the compiler may
% have to be run twice to clear warning/error messages.


\usepackage{hyperref}




% *** CITATION PACKAGES ***
%
\ifCLASSOPTIONcompsoc
  % IEEE Computer Society needs nocompress option
  % requires cite.sty v4.0 or later (November 2003)
  \usepackage[nocompress]{cite}
\else
  % normal IEEE
  \usepackage{cite}
\fi
% cite.sty was written by Donald Arseneau
% V1.6 and later of IEEEtran pre-defines the format of the cite.sty package
% \cite{} output to follow that of the IEEE. Loading the cite package will
% result in citation numbers being automatically sorted and properly
% "compressed/ranged". e.g., [1], [9], [2], [7], [5], [6] without using
% cite.sty will become [1], [2], [5]--[7], [9] using cite.sty. cite.sty's
% \cite will automatically add leading space, if needed. Use cite.sty's
% noadjust option (cite.sty V3.8 and later) if you want to turn this off
% such as if a citation ever needs to be enclosed in parenthesis.
% cite.sty is already installed on most LaTeX systems. Be sure and use
% version 5.0 (2009-03-20) and later if using hyperref.sty.
% The latest version can be obtained at:
% http://www.ctan.org/pkg/cite
% The documentation is contained in the cite.sty file itself.
%
% Note that some packages require special options to format as the Computer
% Society requires. In particular, Computer Society  papers do not use
% compressed citation ranges as is done in typical IEEE papers
% (e.g., [1]-[4]). Instead, they list every citation separately in order
% (e.g., [1], [2], [3], [4]). To get the latter we need to load the cite
% package with the nocompress option which is supported by cite.sty v4.0
% and later. Note also the use of a CLASSOPTION conditional provided by
% IEEEtran.cls V1.7 and later.





% *** GRAPHICS RELATED PACKAGES ***
%
\ifCLASSINFOpdf
  \usepackage[pdftex]{graphicx}
  % declare the path(s) where your graphic files are
  % \graphicspath{{../pdf/}{../jpeg/}}
  % and their extensions so you won't have to specify these with
  % every instance of \includegraphics
  % \DeclareGraphicsExtensions{.pdf,.jpeg,.png}
\else
  % or other class option (dvipsone, dvipdf, if not using dvips). graphicx
  % will default to the driver specified in the system graphics.cfg if no
  % driver is specified.
  % \usepackage[dvips]{graphicx}
  % declare the path(s) where your graphic files are
  % \graphicspath{{../eps/}}
  % and their extensions so you won't have to specify these with
  % every instance of \includegraphics
  % \DeclareGraphicsExtensions{.eps}
\fi
% graphicx was written by David Carlisle and Sebastian Rahtz. It is
% required if you want graphics, photos, etc. graphicx.sty is already
% installed on most LaTeX systems. The latest version and documentation
% can be obtained at: 
% http://www.ctan.org/pkg/graphicx
% Another good source of documentation is "Using Imported Graphics in
% LaTeX2e" by Keith Reckdahl which can be found at:
% http://www.ctan.org/pkg/epslatex
%
% latex, and pdflatex in dvi mode, support graphics in encapsulated
% postscript (.eps) format. pdflatex in pdf mode supports graphics
% in .pdf, .jpeg, .png and .mps (metapost) formats. Users should ensure
% that all non-photo figures use a vector format (.eps, .pdf, .mps) and
% not a bitmapped formats (.jpeg, .png). The IEEE frowns on bitmapped formats
% which can result in "jaggedy"/blurry rendering of lines and letters as
% well as large increases in file sizes.
%
% You can find documentation about the pdfTeX application at:
% http://www.tug.org/applications/pdftex






% *** MATH PACKAGES ***
%
\usepackage{amsmath}
% A popular package from the American Mathematical Society that provides
% many useful and powerful commands for dealing with mathematics.
%
% Note that the amsmath package sets \interdisplaylinepenalty to 10000
% thus preventing page breaks from occurring within multiline equations. Use:
%\interdisplaylinepenalty=2500
% after loading amsmath to restore such page breaks as IEEEtran.cls normally
% does. amsmath.sty is already installed on most LaTeX systems. The latest
% version and documentation can be obtained at:
% http://www.ctan.org/pkg/amsmath





% *** SPECIALIZED LIST PACKAGES ***
%
%\usepackage{algorithmic}
% algorithmic.sty was written by Peter Williams and Rogerio Brito.
% This package provides an algorithmic environment fo describing algorithms.
% You can use the algorithmic environment in-text or within a figure
% environment to provide for a floating algorithm. Do NOT use the algorithm
% floating environment provided by algorithm.sty (by the same authors) or
% algorithm2e.sty (by Christophe Fiorio) as the IEEE does not use dedicated
% algorithm float types and packages that provide these will not provide
% correct IEEE style captions. The latest version and documentation of
% algorithmic.sty can be obtained at:
% http://www.ctan.org/pkg/algorithms
% Also of interest may be the (relatively newer and more customizable)
% algorithmicx.sty package by Szasz Janos:
% http://www.ctan.org/pkg/algorithmicx




% *** ALIGNMENT PACKAGES ***
%
%\usepackage{array}
% Frank Mittelbach's and David Carlisle's array.sty patches and improves
% the standard LaTeX2e array and tabular environments to provide better
% appearance and additional user controls. As the default LaTeX2e table
% generation code is lacking to the point of almost being broken with
% respect to the quality of the end results, all users are strongly
% advised to use an enhanced (at the very least that provided by array.sty)
% set of table tools. array.sty is already installed on most systems. The
% latest version and documentation can be obtained at:
% http://www.ctan.org/pkg/array


% IEEEtran contains the IEEEeqnarray family of commands that can be used to
% generate multiline equations as well as matrices, tables, etc., of high
% quality.




% *** SUBFIGURE PACKAGES ***
%\ifCLASSOPTIONcompsoc
%  \usepackage[caption=false,font=footnotesize,labelfont=sf,textfont=sf]{subfig}
%\else
%  \usepackage[caption=false,font=footnotesize]{subfig}
%\fi
% subfig.sty, written by Steven Douglas Cochran, is the modern replacement
% for subfigure.sty, the latter of which is no longer maintained and is
% incompatible with some LaTeX packages including fixltx2e. However,
% subfig.sty requires and automatically loads Axel Sommerfeldt's caption.sty
% which will override IEEEtran.cls' handling of captions and this will result
% in non-IEEE style figure/table captions. To prevent this problem, be sure
% and invoke subfig.sty's "caption=false" package option (available since
% subfig.sty version 1.3, 2005/06/28) as this is will preserve IEEEtran.cls
% handling of captions.
% Note that the Computer Society format requires a sans serif font rather
% than the serif font used in traditional IEEE formatting and thus the need
% to invoke different subfig.sty package options depending on whether
% compsoc mode has been enabled.
%
% The latest version and documentation of subfig.sty can be obtained at:
% http://www.ctan.org/pkg/subfig




% *** FLOAT PACKAGES ***
%
%\usepackage{fixltx2e}
% fixltx2e, the successor to the earlier fix2col.sty, was written by
% Frank Mittelbach and David Carlisle. This package corrects a few problems
% in the LaTeX2e kernel, the most notable of which is that in current
% LaTeX2e releases, the ordering of single and double column floats is not
% guaranteed to be preserved. Thus, an unpatched LaTeX2e can allow a
% single column figure to be placed prior to an earlier double column
% figure.
% Be aware that LaTeX2e kernels dated 2015 and later have fixltx2e.sty's
% corrections already built into the system in which case a warning will
% be issued if an attempt is made to load fixltx2e.sty as it is no longer
% needed.
% The latest version and documentation can be found at:
% http://www.ctan.org/pkg/fixltx2e


%\usepackage{stfloats}
% stfloats.sty was written by Sigitas Tolusis. This package gives LaTeX2e
% the ability to do double column floats at the bottom of the page as well
% as the top. (e.g., "\begin{figure*}[!b]" is not normally possible in
% LaTeX2e). It also provides a command:
%\fnbelowfloat
% to enable the placement of footnotes below bottom floats (the standard
% LaTeX2e kernel puts them above bottom floats). This is an invasive package
% which rewrites many portions of the LaTeX2e float routines. It may not work
% with other packages that modify the LaTeX2e float routines. The latest
% version and documentation can be obtained at:
% http://www.ctan.org/pkg/stfloats
% Do not use the stfloats baselinefloat ability as the IEEE does not allow
% \baselineskip to stretch. Authors submitting work to the IEEE should note
% that the IEEE rarely uses double column equations and that authors should try
% to avoid such use. Do not be tempted to use the cuted.sty or midfloat.sty
% packages (also by Sigitas Tolusis) as the IEEE does not format its papers in
% such ways.
% Do not attempt to use stfloats with fixltx2e as they are incompatible.
% Instead, use Morten Hogholm'a dblfloatfix which combines the features
% of both fixltx2e and stfloats:
%
% \usepackage{dblfloatfix}
% The latest version can be found at:
% http://www.ctan.org/pkg/dblfloatfix




%\ifCLASSOPTIONcaptionsoff
%  \usepackage[nomarkers]{endfloat}
% \let\MYoriglatexcaption\caption
% \renewcommand{\caption}[2][\relax]{\MYoriglatexcaption[#2]{#2}}
%\fi
% endfloat.sty was written by James Darrell McCauley, Jeff Goldberg and 
% Axel Sommerfeldt. This package may be useful when used in conjunction with 
% IEEEtran.cls'  captionsoff option. Some IEEE journals/societies require that
% submissions have lists of figures/tables at the end of the paper and that
% figures/tables without any captions are placed on a page by themselves at
% the end of the document. If needed, the draftcls IEEEtran class option or
% \CLASSINPUTbaselinestretch interface can be used to increase the line
% spacing as well. Be sure and use the nomarkers option of endfloat to
% prevent endfloat from "marking" where the figures would have been placed
% in the text. The two hack lines of code above are a slight modification of
% that suggested by in the endfloat docs (section 8.4.1) to ensure that
% the full captions always appear in the list of figures/tables - even if
% the user used the short optional argument of \caption[]{}.
% IEEE papers do not typically make use of \caption[]'s optional argument,
% so this should not be an issue. A similar trick can be used to disable
% captions of packages such as subfig.sty that lack options to turn off
% the subcaptions:
% For subfig.sty:
% \let\MYorigsubfloat\subfloat
% \renewcommand{\subfloat}[2][\relax]{\MYorigsubfloat[]{#2}}
% However, the above trick will not work if both optional arguments of
% the \subfloat command are used. Furthermore, there needs to be a
% description of each subfigure *somewhere* and endfloat does not add
% subfigure captions to its list of figures. Thus, the best approach is to
% avoid the use of subfigure captions (many IEEE journals avoid them anyway)
% and instead reference/explain all the subfigures within the main caption.
% The latest version of endfloat.sty and its documentation can obtained at:
% http://www.ctan.org/pkg/endfloat
%
% The IEEEtran \ifCLASSOPTIONcaptionsoff conditional can also be used
% later in the document, say, to conditionally put the References on a 
% page by themselves.




% *** PDF, URL AND HYPERLINK PACKAGES ***
%
%\usepackage{url}
% url.sty was written by Donald Arseneau. It provides better support for
% handling and breaking URLs. url.sty is already installed on most LaTeX
% systems. The latest version and documentation can be obtained at:
% http://www.ctan.org/pkg/url
% Basically, \url{my_url_here}.





% *** Do not adjust lengths that control margins, column widths, etc. ***
% *** Do not use packages that alter fonts (such as pslatex).         ***
% There should be no need to do such things with IEEEtran.cls V1.6 and later.
% (Unless specifically asked to do so by the journal or conference you plan
% to submit to, of course. )


% correct bad hyphenation here
\hyphenation{op-tical net-works semi-conduc-tor}


\begin{document}
%
% paper title
% Titles are generally capitalized except for words such as a, an, and, as,
% at, but, by, for, in, nor, of, on, or, the, to and up, which are usually
% not capitalized unless they are the first or last word of the title.
% Linebreaks \\ can be used within to get better formatting as desired.
% Do not put math or special symbols in the title.
\title{Orchestration Load Indicators and Patterns:\\In-the-wild Studies Using Mobile Eye-tracking}
%
%
% author names and IEEE memberships
% note positions of commas and nonbreaking spaces ( ~ ) LaTeX will not break
% a structure at a ~ so this keeps an author's name from being broken across
% two lines.
% use \thanks{} to gain access to the first footnote area
% a separate \thanks must be used for each paragraph as LaTeX2e's \thanks
% was not built to handle multiple paragraphs
%
%
%\IEEEcompsocitemizethanks is a special \thanks that produces the bulleted
% lists the Computer Society journals use for "first footnote" author
% affiliations. Use \IEEEcompsocthanksitem which works much like \item
% for each affiliation group. When not in compsoc mode,
% \IEEEcompsocitemizethanks becomes like \thanks and
% \IEEEcompsocthanksitem becomes a line break with idention. This
% facilitates dual compilation, although admittedly the differences in the
% desired content of \author between the different types of papers makes a
% one-size-fits-all approach a daunting prospect. For instance, compsoc 
% journal papers have the author affiliations above the "Manuscript
% received ..."  text while in non-compsoc journals this is reversed. Sigh.

\author{Luis~P.~Prieto,~\IEEEmembership{Member,~IEEE,}
        Kshitij~Sharma,
        {\L}ukasz~Kidzinski,
        and~Pierre~Dillenbourg% <-this % stops a space
\IEEEcompsocitemizethanks{\IEEEcompsocthanksitem All co-authors were with the Computer Human Interaction for Learning and Instruction Lab (CHILI), \'Ecole Polytechnique F\'ed\'erale de Lausanne, Switzerland.\protect\\
% note need leading \protect in front of \\ to get a newline within \thanks as
% \\ is fragile and will error, could use \hfil\break instead.
E-mail: see https://people.epfl.ch/luis.prieto
}% <-this % stops an unwanted space
\thanks{Manuscript received April XX, XXXX; revised August XX, XXXX.}}

% note the % following the last \IEEEmembership and also \thanks - 
% these prevent an unwanted space from occurring between the last author name
% and the end of the author line. i.e., if you had this:
% 
% \author{....lastname \thanks{...} \thanks{...} }
%                     ^------------^------------^----Do not want these spaces!
%
% a space would be appended to the last name and could cause every name on that
% line to be shifted left slightly. This is one of those "LaTeX things". For
% instance, "\textbf{A} \textbf{B}" will typeset as "A B" not "AB". To get
% "AB" then you have to do: "\textbf{A}\textbf{B}"
% \thanks is no different in this regard, so shield the last } of each \thanks
% that ends a line with a % and do not let a space in before the next \thanks.
% Spaces after \IEEEmembership other than the last one are OK (and needed) as
% you are supposed to have spaces between the names. For what it is worth,
% this is a minor point as most people would not even notice if the said evil
% space somehow managed to creep in.



% The paper headers
\markboth{IEEE Transactions on Learning Technologies,~Vol.~X, No.~X, August~XXXX}%
{Prieto \MakeLowercase{\textit{et al.}}: Orchestration Load Indicators and Patterns}
% The only time the second header will appear is for the odd numbered pages
% after the title page when using the twoside option.
% 
% *** Note that you probably will NOT want to include the author's ***
% *** name in the headers of peer review papers.                   ***
% You can use \ifCLASSOPTIONpeerreview for conditional compilation here if
% you desire.



% The publisher's ID mark at the bottom of the page is less important with
% Computer Society journal papers as those publications place the marks
% outside of the main text columns and, therefore, unlike regular IEEE
% journals, the available text space is not reduced by their presence.
% If you want to put a publisher's ID mark on the page you can do it like
% this:
%\IEEEpubid{0000--0000/00\$00.00~\copyright~2015 IEEE}
% or like this to get the Computer Society new two part style.
%\IEEEpubid{\makebox[\columnwidth]{\hfill 0000--0000/00/\$00.00~\copyright~2015 IEEE}%
%\hspace{\columnsep}\makebox[\columnwidth]{Published by the IEEE Computer Society\hfill}}
% Remember, if you use this you must call \IEEEpubidadjcol in the second
% column for its text to clear the IEEEpubid mark (Computer Society jorunal
% papers don't need this extra clearance.)



% use for special paper notices
%\IEEEspecialpapernotice{(Invited Paper)}



% for Computer Society papers, we must declare the abstract and index terms
% PRIOR to the title within the \IEEEtitleabstractindextext IEEEtran
% command as these need to go into the title area created by \maketitle.
% As a general rule, do not put math, special symbols or citations
% in the abstract or keywords.
\IEEEtitleabstractindextext{%
\begin{abstract}
%100-200 words
%\boldmath
%\blindtext[1]
Orchestration load (the effort a teacher spends in coordinating the multiple activities and learning processes in an educational setting) has been proposed as a helpful construct in designing educational technologies that not only support learning, but also are usable at the classroom level. However, so far this notion has remained fuzzy, used solely in a high-level and abstract way. In order to ground orchestration load in empirical evidence and study it in a more systematic and detailed manner, we propose a model of factors that affect such load, and a measure for moment-to-moment orchestration load based on physiological data (concretely, mobile eye-tracking measures), along with behavioral data. This paper presents the results of applying this method to four exploratory case studies, where teachers orchestrated technology-enhanced face-to-face lessons with primary, secondary school and university students. The data from these studies provides a first validation of this method in different conditions, and illustrate how it can be used to understand the effect of different classroom factors on the orchestration load of a teacher, and to extract evidence-based insights about classroom orchestration with technology.
\end{abstract}

% Note that keywords are not normally used for peerreview papers.
\begin{IEEEkeywords}
Orchestration load, Eye-tracking, Cognitive load, Classroom studies.
\end{IEEEkeywords}}


% make the title area
\maketitle


% To allow for easy dual compilation without having to reenter the
% abstract/keywords data, the \IEEEtitleabstractindextext text will
% not be used in maketitle, but will appear (i.e., to be "transported")
% here as \IEEEdisplaynontitleabstractindextext when the compsoc 
% or transmag modes are not selected <OR> if conference mode is selected 
% - because all conference papers position the abstract like regular
% papers do.
\IEEEdisplaynontitleabstractindextext
% \IEEEdisplaynontitleabstractindextext has no effect when using
% compsoc or transmag under a non-conference mode.



% For peer review papers, you can put extra information on the cover
% page as needed:
% \ifCLASSOPTIONpeerreview
% \begin{center} \bfseries EDICS Category: 3-BBND \end{center}
% \fi
%
% For peerreview papers, this IEEEtran command inserts a page break and
% creates the second title. It will be ignored for other modes.
\IEEEpeerreviewmaketitle



\IEEEraisesectionheading{\section{Introduction}\label{sec:introduction}}
% Computer Society journal (but not conference!) papers do something unusual
% with the very first section heading (almost always called "Introduction").
% They place it ABOVE the main text! IEEEtran.cls does not automatically do
% this for you, but you can achieve this effect with the provided
% \IEEEraisesectionheading{} command. Note the need to keep any \label that
% is to refer to the section immediately after \section in the above as
% \IEEEraisesectionheading puts \section within a raised box.

% The very first letter is a 2 line initial drop letter followed
% by the rest of the first word in caps (small caps for compsoc).
% 
% form to use if the first word consists of a single letter:
% \IEEEPARstart{A}{demo} file is ....
% 
% form to use if you need the single drop letter followed by
% normal text (unknown if ever used by the IEEE):
% \IEEEPARstart{A}{}demo file is ....
% 
% Some journals put the first two words in caps:
% \IEEEPARstart{T}{his demo} file is ....
% 
% Here we have the typical use of a "T" for an initial drop letter
% and "HIS" in caps to complete the first word.

%\IEEEPARstart{T}{his} demo file is intended to serve as a ``starter file'' ...

\begin{itemize}
\item Teacher facilitation is a crucial factor in TEL in authentic, face-to-face settings \cite{Gomez2013,Onrubia2012}, often under the term `orchestration' (definition from \cite{Dillenbourg2009})
\item The term is used in a variety of meanings  \cite{Prieto2011}, but in learning technologies (LT) research, it specifically addresses the challenges of TEL practice in \textit{authentic} settings \cite{Roschelle2013}
\item From the educational technology designer perspective, orchestration-related research denotes a focus on `usability at the classroom level' \cite{Dillenbourg2011}
\item In this sense, LT researchers speak of `orchestration load' \cite{Dillenbourg2013,Cuendet2013,munoz2013sharing}, in analogy to cognitive load, a concept studied thoroughly in cognitive science, educational psychology and human-computer interaction \cite{sweller1994cognitive,oviatt2006human}
\item However, in contrast with cognitive load (studied in controlled laboratory settings), orchestration load needs to be studied in authentic conditions. This has led to the term being used in high-level, abstract terms \cite{Dillenbourg2013,Cuendet2013}, seldom quantified except through ad-hoc proxies such as classroom workflow efficiency \cite{Alavi2012}
\item In our previous work we have explored a mixed quanti/quali approach to studying orchestration load in authentic settings in a more concrete and detailed manner \cite{Prieto2015ectel}. However, we still lack models of orchestration load, and reliable measures to help us compare different classroom situations in terms of orchestration load
\item This paper presents one such model and measure, and their initial validation through several case studies in a variety of authentic educational settings, from primary school students using tabletop technologies to university students with laptops
\item The rest of the paper is structured as follows...
\end{itemize}


% needed in second column of first page if using \IEEEpubid
%\IEEEpubidadjcol



% An example of a floating figure using the graphicx package.
% Note that \label must occur AFTER (or within) \caption.
% For figures, \caption should occur after the \includegraphics.
% Note that IEEEtran v1.7 and later has special internal code that
% is designed to preserve the operation of \label within \caption
% even when the captionsoff option is in effect. However, because
% of issues like this, it may be the safest practice to put all your
% \label just after \caption rather than within \caption{}.
%
% Reminder: the "draftcls" or "draftclsnofoot", not "draft", class
% option should be used if it is desired that the figures are to be
% displayed while in draft mode.
%
%\begin{figure}[!t]
%\centering
%\includegraphics[width=2.5in]{myfigure}
% where an .eps filename suffix will be assumed under latex, 
% and a .pdf suffix will be assumed for pdflatex; or what has been declared
% via \DeclareGraphicsExtensions.
%\caption{Simulation results for the network.}
%\label{fig_sim}
%\end{figure}

% Note that the IEEE typically puts floats only at the top, even when this
% results in a large percentage of a column being occupied by floats.
% However, the Computer Society has been known to put floats at the bottom.


% An example of a double column floating figure using two subfigures.
% (The subfig.sty package must be loaded for this to work.)
% The subfigure \label commands are set within each subfloat command,
% and the \label for the overall figure must come after \caption.
% \hfil is used as a separator to get equal spacing.
% Watch out that the combined width of all the subfigures on a 
% line do not exceed the text width or a line break will occur.
%
%\begin{figure*}[!t]
%\centering
%\subfloat[Case I]{\includegraphics[width=2.5in]{box}%
%\label{fig_first_case}}
%\hfil
%\subfloat[Case II]{\includegraphics[width=2.5in]{box}%
%\label{fig_second_case}}
%\caption{Simulation results for the network.}
%\label{fig_sim}
%\end{figure*}
%
% Note that often IEEE papers with subfigures do not employ subfigure
% captions (using the optional argument to \subfloat[]), but instead will
% reference/describe all of them (a), (b), etc., within the main caption.
% Be aware that for subfig.sty to generate the (a), (b), etc., subfigure
% labels, the optional argument to \subfloat must be present. If a
% subcaption is not desired, just leave its contents blank,
% e.g., \subfloat[].


% An example of a floating table. Note that, for IEEE style tables, the
% \caption command should come BEFORE the table and, given that table
% captions serve much like titles, are usually capitalized except for words
% such as a, an, and, as, at, but, by, for, in, nor, of, on, or, the, to
% and up, which are usually not capitalized unless they are the first or
% last word of the caption. Table text will default to \footnotesize as
% the IEEE normally uses this smaller font for tables.
% The \label must come after \caption as always.
%
%\begin{table}[!t]
%% increase table row spacing, adjust to taste
%\renewcommand{\arraystretch}{1.3}
% if using array.sty, it might be a good idea to tweak the value of
% \extrarowheight as needed to properly center the text within the cells
%\caption{An Example of a Table}
%\label{table_example}
%\centering
%% Some packages, such as MDW tools, offer better commands for making tables
%% than the plain LaTeX2e tabular which is used here.
%\begin{tabular}{|c||c|}
%\hline
%One & Two\\
%\hline
%Three & Four\\
%\hline
%\end{tabular}
%\end{table}


% Note that the IEEE does not put floats in the very first column
% - or typically anywhere on the first page for that matter. Also,
% in-text middle ("here") positioning is typically not used, but it
% is allowed and encouraged for Computer Society conferences (but
% not Computer Society journals). Most IEEE journals/conferences use
% top floats exclusively. 
% Note that, LaTeX2e, unlike IEEE journals/conferences, places
% footnotes above bottom floats. This can be corrected via the
% \fnbelowfloat command of the stfloats package.

\section{Classroom Orchestration and Cognitive Load Measures}\label{sec:related}

\begin{itemize}
\item Face-to-face classroom management is important for learning outcomes but also is highly demanding (public space, multiple activities, immediacy, unpredictability) \cite{Doyle2006}
\item Technology adds another layer of complexity to the classroom, which may explain reluctance for adoption of technology, and focus of LT research on 'orchestration' as 'classroom usability' \cite{Dillenbourg2011}
\item By testing LTs in authentic classrooms and observing what works, researchers are starting to come up with guidelines of orchestrable technology \cite{Cuendet2013,Dillenbourg2013,Kharrufa2013,Kreitmayer2013}
\item This approach to classroom usability reminds of the beginnings of usability studies \cite{Webusability} (focus on a few users finishing the task -- ``teacher heroes'' \cite{Dillenbourg2009b})
\item Another parallel with usability: the use of cognitive load \cite{Paas2004}, or in this case `orchestration load' (``the effort necessary for the teacher -- and other actors -- to conduct learning activities'' \cite{Cuendet2013}) \cite{Dillenbourg2013}
\item To advance (classroom) usability beyond this artisanal/observational approach, usable (classroom) computer systems require 1) focus on users; 2) iterative design and testing; and 3) empirical measurement of usage \cite{Gould1985}
\item The first two are already a part of much of LT research. But the third one is made difficult by the simultaneity and immediacy of face-to-face classroom activities. We need empirical evidence of how the orchestration process unfolds!
\item To do this, we could use the wealth of methods in psychology and HCI to measure cognitive load \cite{Brunken2003} -- as long as we stay within \textit{authentic} educational settings (inherent to classroom orchestration \cite{Roschelle2013})
\item No single method is accepted as valid for every task, and measures are often taken in controlled lab conditions, with simple tasks. Given this lack of control, the triangulation of different methods might be needed. An example is \cite{Buettner2013}, which used four different eye-tracking measurements to track cognitive load in a learning task.
\item In order to be able to measure orchestration empirically, we also need a model of orchestration load that help us start disentangling this multi-tasking, multi-target activity 
\end{itemize}

\section{A Model of Factors in Run-time Classroom Orchestration}
\label{sec:model}
\begin{itemize}
\item Classroom management/orchestration is assumed to be multi-tasking \cite{Doyle2006} in the sense of multiple constraints and concerns, but most often the orchestration is a \textit{sequence} of actions and decisions (put an example)
\item From the definition of orchestration \cite{Dillenbourg2009}, we can see there are several \textit{process variables} that can affect the orchestration load at any point in time, but which may not be interesting for us when designing the technology: what is the current \textit{coordinating activity}, at which \textit{social plane} is the coordination happening, or what is the \textit{resource} on which the activity is focusing
\item The load of each process variable may vary from teacher to teacher (experience, personality, training...). Hence, to have an idea of the load experienced by the teacher we have to take out somehow these influences
\item Hence, as a first approximation to the instantaneous orchestration load, we could pose the following linear model:
\begin{equation}
OL = \beta_0 + \beta_1*A + \beta_2*S + \beta_3*F + \epsilon
\label{formOLScat}
\end{equation}
%TODO: Did we finally use a linear model or a GAM?
\item ...where A (current coordination activity) can be Explanation, Monitoring, Questioning etc. (i.e., for an individual teacher, different activities may represent different amounts of load); similarly, S (social plane) can be Individual, Small group, Class-wide; and F (focus of activity) can be the different classroom elements on which the teacher may focus, including students (could be represented graphically as an orchestration graph, see \cite{prieto2011recurrent}). 
\item If we transform the categorical variables in Formula \ref{formOLScat} into binary variables, we are left with:
\begin{multline}
\label{formOLS}
OL = \beta_0 + \beta_{11}*A_{expl} + \beta_{12}*A_{monit} + ... \\ + \beta_{21}*S_{indiv} + ... + \beta_{31}*F_{stud} + ... + \epsilon
\end{multline}
\item In this model, as designers of technology, we are interested on mostly on measuring the value of $\epsilon$ (the orchestration load of the situation once we take away the effect of the known process variables), which is the one that our technology might affect (the other being mostly due to the teacher peculiarities and the classroom occurrences out of our control)
\end{itemize}


\begin{figure}[!t]
\centering
\includegraphics[width=\linewidth]{img/ModelFactorsOL}
\caption{A model of factors influencing orchestration load}
\label{fig:model}
\end{figure}


\section{Estimating Orchestration Load Using Physiological (Eye-tracking) and Behavioral Measures}
\label{sec:measures}
\begin{itemize}
\item At the most basic level, we want to track orchestration load (OL) in a moment-to-moment manner, in order to later take away the influence of process variables and find out if two situations had distinctly lower/higher load
\item Since direct measurement of load (e.g., with dual-task) is problematic in complex multi-tasking activities \cite{Paas2003}, we have to resort to indirect measures (e.g., physiological)
\item To estimate the influence of the process variables (teaching activity, social plane, focus of gaze) on load, we can resort to behavioral measures (i.e., observation of the classroom occurrences) -- e.g., through manual video coding
\item Mobile eye-tracking can be used to obtain both relevant physiological data about cognitive load \cite{Buettner2013} and audiovisual feed for teacher behavior analysis without additional equipment (crucial when investigating in authentic settings). Indeed, we have shown the feasibility of this approach in explorations of orchestration load in the past \cite{Prieto2014,Prieto2015cscl,Prieto2015ectel}
\item We propose the following method for analysis combining behavioral and physiological measures (see Figure \ref{fig:analysis}):
\begin{enumerate}
\item Record eye-tracking of the session
\item Aggregate the four eye-tracking measures related to cognitive load, in 10s episodes\footnote{10s was chosen because it is short enough to capture variations in eye-tracking measures, but long enough so that can be coded by a researcher in terms of behavior (what the teacher is doing) in a meaningful manner}
\item Calculate the first Principal Component (in PCA) of the four eye-tracking signals. The score of each 10s episode along this dimension, we will call Orchestration Load Score (OLS, our best approximation to the OLS in formula \ref{formOLS}) %TODO: insert here the Lukasz transform, if applied
\item Select the "extreme load episodes"\footnote{This is done to keep the manual researcher effort under control, but the whole session videos could be video coded too.} of the session (the episodes where all four measures agree on being higher or lower than the session median), and code them manually in terms of the process variables (Activity, Social plane, Focus of gaze)
\item Build statistical models of the orchestration load, e.g.,  similar to the linear one in formula \ref{formOLS} (see section \ref{sec:study4} to see an example)
\end{enumerate}
\end{itemize}
%TODO: add footnote about the sign issue, and how we solve it by positive signs and/or similarity with our PCA loadings (internal product)


\begin{figure}[!t]
\centering
\includegraphics[width=\linewidth]{img/AnalysisMethodBase.png}
\caption{Diagram representing the physiological-behavioral analysis of orchestration load}
\label{fig:analysis}
\end{figure}

\section{Case Studies}

\subsection{Methodology}

\begin{itemize}
\item The goals (research questions) of this paper are twofold:
\begin{enumerate}
\item Is the aforementioned PCA-based orchestration load score (OLS), and the proposed statistical models, able to discriminate the load of  substantially different situations in terms of orchestration load?
\item Can the application of these measures and models provide us with insights about the orchestration load of a certain classroom situation?
\end{enumerate}
\item We have used the aforementioned approach and models for orchestration load in four case studies, spanning a total of 14 sessions
orchestrated by four different teachers with actual students at different educational levels, from 11yrs old to university students (see Table \ref{tab:cases})
\item The first three case studies aimed at validating the aforementioned model and physiological-behavioral measures (RQ1). In each of these validation studies we compared two orchestration situations which were similar in most aspects, but varied significantly in one aspect which we can assume is directly related to the teacher orchestration load (teacher expertise, familiarity with the classroom technology, having additional help from a fellow orchestrator). 
\item While it is debatable that the situations had different inherent loads \textit{at all moments}, it is as much as we can manipulate and make comparations between inherently unique situations, while maintaining authenticity
\item Finally, the fourth case study was not manipulated, and is provided as an illustrative example of how the models and methods can be used to extract insights about the orchestration load of a certain classroom/situation (RQ2)
\item Initial explorations of the datasets from these cases have been published before for cases 1 and 4 \cite{Prieto2015cscl}, and for the
first two sessions of case 2 \cite{Prieto2015ectel}. However, the application of the statistical models and methods depicted in sections \ref{sec:model} and \ref{sec:measures} is completely new
\item Below, we describe the context and main results of the case studies. The datasets generated, full analytical code and detailed results are available online\footnote{\href{https://github.com/chili-epfl/paper-IEEETLT-orchestrationload}{https://github.com/chili-epfl/paper-IEEETLT-orchestrationload}.}
\item In all 4 studies the data gathering and analysis method of section \ref{sec:measures} was followed: a) eyetracking data was recorded and cognitive load-related metrics were aggregated for 10s episodes; b) Principal Component Analysis was performed on the data from the four cases (see Figure \ref{fig:pca}) to obtain the OLS (1st component score); c) "Extreme load episodes" were video coded by a single researcher, to extract the process variables (activity, social plane, focus of gaze) 
\end{itemize}

\begin{figure}[!t]
\centering
\includegraphics[width=\linewidth]{img/PCA.png}
\caption{Projection of the four eye-tracking measures recorded throughout the four case studies, on the space defined by the first two PCA components. The Orchestration Load Score (OLS) is represented by the horizontal axis of the figure}
\label{fig:pca}
\end{figure}




\begin{table*}[!t]
%% increase table row spacing, adjust to taste
%\renewcommand{\arraystretch}{1.3}
% if using array.sty, it might be a good idea to tweak the value of
% \extrarowheight as needed to properly center the text within the cells
\caption{Summary of main case study characteristics}
\label{tab:cases}
\centering
%% Some packages, such as MDW tools, offer better commands for making tables
%% than the plain LaTeX2e tabular which is used here.
\begin{tabular}{|c||p{1.5cm}|p{1.5cm}|p{2cm}|p{3.4cm}|p{2cm}|p{3.4cm}|}
\hline
Study & Setting & Teachers & Sessions (session length) & Technological support & Main goal & Target variable\\
\hline
\hline
Case 1 & University & 1 expert, 1 novice & 2+1 (45-65' each) & Laptops, classroom projector & Validate measure/model & Teacher expertise (novice vs. expert) \\
\hline
Case 2 & Primary school & 1 expert & 2+2 (80' each) & Laptops, classroom projector // Tabletops, classroom projector & Validate measure/model & Familiarity with technology (usual vs. novel) \\
\hline
Case 3 & Open doors (primary) & 1 novice (researcher) & 4 (35-45' each) & Tabletops, classroom projector & Validate measure/model & External (human) help (without/with helper) \\
\hline
Case 4 & Open doors (primary) & 1 novice (researcher) & 3 (35-45' each) & Tabletops & Illustrate use & -- \\
\hline
\end{tabular}
\end{table*}



\subsection{Study 1: Exploring Teacher Expertise in an University Course}

\begin{itemize}
\item One of the most obvious factors in orchestration load is teacher expertise: more expert teachers have internalized/automated many aspects of their teaching \cite{prieto2011recurrent,feldon2007cognitive} will manage the classroom with more ease (less load) than novice ones
\item Case study question: \textit{Will our model and OL score be able to discriminate similar learning situations, orchestrated by a novel and an experienced teacher?}
\end{itemize}

\subsubsection{Context}

\begin{itemize}
\item Face-to-face master-level course on ``Digital education and learning analytics'' at EPFL. Classes had 10-12 students, working with laptops and teacher using laptop, projector and whiteboard (see Figure \ref{fig:case1picture}). The lessons were a fluid combination of lecture, questioning student and exercises
\item Two lessons with the same student cohort were recorded for the expert teacher, and one for the novice teacher, 45-65min each (in reality, two sessions were recorded for each, but one of the novice sessions had to be thrown out due to technical problems during data gathering)
\end{itemize}


\begin{figure}[!t]
\centering
\includegraphics[width=\linewidth]{img/Case1Picture}
\caption{Classroom setup during Case study 1 (master-level course)}
\label{fig:case1picture}
\end{figure}

\subsubsection{Results}

\begin{itemize}
\item To validate whether the OLS is a good predictor for orchestration load in this context, we assume as a ground truth that the lessons orchestrated by the novice teacher represented a higher load (Load=1) than the ones orchestrated by the expert (Load=0). 
\item We then train a logistic regression model to try to predict the aforementioned ``ground truth'' using the known process variables (activity, social, focus -- coded from the video by a researcher) and the OLS
\item As we can see in table \ref{tab:case1results}, once we take away the influence of the process variables, the OLS still remains a significant predictor of the assumed (binary) orchestration load ($p<0.001$, see table \ref{tab:case1results}). Furthermore, this model including the OLS explains a considerable amount of the variance in the data (McFadden's pseudo-$R^2=0.88$) 
\item An analysis of variance of the different variables in the model confirms this, with the OLS explaining most of the deviance in the data ($p<0.001$). 
\end{itemize}

\begin{table}[!t]
%% increase table row spacing, adjust to taste
%\renewcommand{\arraystretch}{1.3}
% if using array.sty, it might be a good idea to tweak the value of
% \extrarowheight as needed to properly center the text within the cells
\caption{Logistic regression model to predict assumed orchestration load (1--Novice, 0--Expert teacher) in case study 1}
\label{tab:case1results}
\centering
%% Some packages, such as MDW tools, offer better commands for making tables
%% than the plain LaTeX2e tabular which is used here.
\begin{tabular}{|p{2.8cm}||r|r|r|r|}
\hline
Coefficient & Estimate & Std. error & z & p-value\\
\hline
\hline
Intercept (Explanation, Class-level, Focus on student faces) & -2.93 & 1.52 & -1.93 & 0.05 \\
Activity: Monitoring & 0.28 & 1.46 & 0.19 & 0.84 \\
Activity: Questioning & -1.66 & 1.73 & -0.96 & 0.33 \\
Activity: Repairs & 2.80 & 3.54 & 0.79 & 0.43 \\
Activity: Task distribution & -16.22 & 2765 & -0.01 & 1.00 \\
Social: Individual & 5.76 & 3.23 & 1.78 & 0.07 \\
Focus: Projector & -3.49 & 2.90 & -1.21 & 0.23 \\
Focus: Student computer & 1.00 & 1.72 & 0.58 & 0.56 \\
Focus: Table & 1.77 & 5.96 & 0.30 & 0.77 \\
Focus: Teacher computer & -2.04 & 1.40 & -1.45 & 0.15 \\
Focus: Whiteboard & -2.91 & 2.78 & -1.05 & 0.29 \\
\textbf{OLS} & \textbf{8.42} & \textbf{2.30} & \textbf{3.67} & \textbf{0.0002***} \\
\hline
\end{tabular}
\end{table}


\subsection{Study 2: Exploring Familiarity with Technology in a Primary School Classroom}

\begin{itemize}
\item Another factor that can affect orchestration load in a predictable manner is familiarity with the technology: a teacher using a new tool will not have developed the necessary automatisms to effortlessly use it in the classroom \cite{feldon2007cognitive}, and hence will experience a higher orchestration load
\item Case study question: \textit{Will our model and OL score be able to discriminate learning situations, orchestrated by a same teacher, but using her usual and novel classroom tools?}
\end{itemize}

\subsubsection{Context}

\begin{itemize}
\item Face-to-face beginning of secondary school (11-12 yrs old), private international Swiss school, experienced teacher (more than 20 yrs experience), maths, two different cohorts of students (18-22 students each) %TODO: Add photos of the usual and novel classrooms -- I need Kshitij to access!
\item Two sessions recorded while teacher orchestrated student work using laptops and a geometry software (Geometer's Sketchpad) as well as a classroom management software; two sessions recorded while teacher orchestrated student groupwork using a novel technology (augmented paper tabletop interfaces, based on Tinkerlamp \cite{do2012tinkerlamp} and public display/projector)
\end{itemize}

\subsubsection{Results}

\begin{itemize}
\item To validate whether the OLS is a good predictor for orchestration load in this context, we assume as a ground truth that the lessons orchestrated using novel technology represented a higher load (Load=1) than the ones orchestrated with the usual technology (Load=0). 
\item We then train a logistic regression model to try to predict the aforementioned ``ground truth'' using the known process variables (activity, social, focus -- coded from the video by a researcher) and the OLS
\item As we can see in table \ref{tab:case2results}, once we take away the influence of the process variables, the OLS still remains a significant predictor of the assumed (binary) orchestration load ($p=0.002$). Furthermore, this model including the OLS explains a considerable amount of the variance in the data (McFadden's pseudo-$R^2=0.75$) 
\item An analysis of variance of the different variables in the model confirms this, with the OLS explaining an appreciable part of the deviance in the data ($p<0.001$). 
\end{itemize}

\begin{table}[!t]
%% increase table row spacing, adjust to taste
%\renewcommand{\arraystretch}{1.3}
% if using array.sty, it might be a good idea to tweak the value of
% \extrarowheight as needed to properly center the text within the cells
\caption{Logistic regression model to predict assumed orchestration load (1--New classroom technology, 0--Usual technology) in case study 2}
\label{tab:case2results}
\centering
%% Some packages, such as MDW tools, offer better commands for making tables
%% than the plain LaTeX2e tabular which is used here.
\begin{tabular}{|p{2.8cm}||r|r|r|r|}
\hline
Coefficient & Estimate & Std. error & z & p-value\\
\hline
\hline
Intercept (Explanation, Class-level, Focus on student faces) & -1.21 & 0.45 & -2.67 & 0.008** \\
Activity: Monitoring & 2.92 & 0.71 & 4.08 & 0.00004*** \\
Activity: Questioning & -1.54 & 0.78 & -1.98 & 0.05* \\
Activity: Repairs & 0.37 & 0.93 & 0.40 & 0.69 \\
Activity: Task distribution & 3.22 & 0.92 & 3.52 & 0.0004*** \\
Social: Small group & 3.93 & 1.11 & 3.53 & 0.0004*** \\
Social: Individual & -0.25 & 0.91 & -0.28 & 0.78 \\
Focus: Student laptop & -20.45 & 1577 & -0.01 & 0.99 \\
Focus: Paper elements & -2.31 & 0.74 & -3.14 & 0.002** \\
Focus: Projector & 20.92 & 2921 & 0.01 & 0.99 \\
Focus: Tabletop computer & 17.39 & 1758 & 0.01 & 0.99 \\
Focus: Teacher computer & -2.68 & 0.92 & -2.93 & 0.003** \\
\textbf{OLS} & \textbf{0.76} & \textbf{0.25} & \textbf{3.04} & \textbf{0.002**} \\
\hline
\end{tabular}
\end{table}


\subsection{Study 3: Exploring External Help in an Open Doors Day}

\begin{itemize}
\item Finally, we can also manipulate orchestration load of a classroom situation by having the teacher have an assistant\footnote{With clear pre-defined roles and task distribution among both facilitators, so that managing the assistant does not generate additional load.}: a teacher with an assistant will experience a lower orchestration load than one that has assistance
\item Case study question: \textit{Will our model and OL score be able to discriminate similar learning situations, orchestrated by a same teacher, with and without an assistant orchestrator?}
\end{itemize}

\subsubsection{Context}

\begin{itemize}
\item Face-to-face math lessons enacted during an open-doors day in the lab with local school children (10-12 yrs old), public Swiss schools, novice teacher that was not the usual teacher (usual teachers present as passive observers), maths lesson combining mini-lecture, groupwork with tabletops and whole-class activity on projector (game), four different cohorts of students (19-21 students each) %TODO: Add photo of the classroom
\item Four sessions recorded while teacher orchestrated the multi-tabletop classroom (augmented paper tabletop interfaces, based on Tinkerlamp \cite{do2012tinkerlamp} and public display/projector); in two of the sessions the assistant helped during the groupwork, dealing with half of the groups (in the rest of the lesson the assistant played no role)
\item To answer the question about OLS as discriminator, only the relevant moments of the lesson (the groupwork) have been taken into account
\end{itemize}

\subsubsection{Results}


\begin{itemize}
\item To validate whether the OLS is a good predictor for orchestration load in this context, we assume as a ground truth that the lessons orchestrated without an assistant represented a higher load (Load=1) than the ones orchestrated with the assistant (Load=0) -- at least, for the parts of the lesson where the assistant had a role 
\item We then train a logistic regression model to try to predict the aforementioned ``ground truth'' using the known process variables (activity, social, focus -- coded from the video by a researcher) and the OLS
\item As we can see in table \ref{tab:case3results}, once we take away the influence of the process variables, the OLS shows clear, adequate trends (positive coefficient), but is no longer a significant predictor of the assumed (binary) orchestration load ($p=0.13$). Furthermore, this model including the OLS explains little variance in the data (McFadden's pseudo-$R^2=0.05$) 
\item This may be due to the fact that we take only (the model is trained with 163 samples, versus $~400$ samples in the previous studies). We probably would have needed more data to see trends more clearly 
\end{itemize}

\begin{table}[!t]
%% increase table row spacing, adjust to taste
%\renewcommand{\arraystretch}{1.3}
% if using array.sty, it might be a good idea to tweak the value of
% \extrarowheight as needed to properly center the text within the cells
\caption{Logistic regression model to predict assumed orchestration load (1--Without assistant, 0--With assistant) in case study 3}
\label{tab:case3results}
\centering
%% Some packages, such as MDW tools, offer better commands for making tables
%% than the plain LaTeX2e tabular which is used here.
\begin{tabular}{|p{2.8cm}||r|r|r|r|}
\hline
Coefficient & Estimate & Std. error & z & p-value\\
\hline
\hline
Intercept (Monitoring, Class-level, Focus on student backs) & 1.55 & 1.13 & 1.37 & 0.16 \\
Activity: Questioning & 0.67 & 0.58 & 1.16 & 0.25 \\
Activity: Repairs & 0.95 & 0.41 & 2.30 & 0.02* \\
Activity: Task distribution & -0.37 & 0.64 & -0.58 & 0.56 \\
Social: Small group & -0.18 & 0.66 & 0.27 & 0.79 \\
Focus: Student faces & -1.65 & 1.27 & -1.30 & 0.19 \\
Focus: Projector & -1.54 & 1.24 & -1.24 & 0.21 \\
Focus: Tabletop computer & -0.77 & 1.24 & -0.62 & 0.54 \\
\textbf{OLS} & \textbf{0.86} & \textbf{0.57} & \textbf{1.51} & \textbf{0.13} \\
\hline
\end{tabular}
\end{table}




\subsection{Study 4: Applying the Measure to an Unmanipulated Setting}
\label{sec:study4}

\subsubsection{Context}

\subsubsection{Results}




\section{Discussion}

\subsection{Eye-tracking Measures to Estimate Orchestration Load}

... the method still requires manual work, but that is about to change \cite{prieto2016teaching}
... the eyetracker is too conspicuous, but this is about to change too as eyetrackers become amost undistinguishable from normal glasses


\subsection{Orchestration Load Patterns and Learning Technology Insights}

Talk about the insights from the linear models of the 4 case studies (do not show full tables, refer to the additional materials) -- maybe just a table with the different teachers and situations, and significant trends?

... technology affecting the mix of activities and hence the OL, without technology being the focus per se

\section{Conclusion}
The conclusion goes here.





% if have a single appendix:
%\appendix[Proof of the Zonklar Equations]
% or
%\appendix  % for no appendix heading
% do not use \section anymore after \appendix, only \section*
% is possibly needed

% use appendices with more than one appendix
% then use \section to start each appendix
% you must declare a \section before using any
% \subsection or using \label (\appendices by itself
% starts a section numbered zero.)
%




% use section* for acknowledgment
\ifCLASSOPTIONcompsoc
  % The Computer Society usually uses the plural form
  \section*{Acknowledgments}
\else
  % regular IEEE prefers the singular form
  \section*{Acknowledgment}
\fi


The authors would like to thank...


% Can use something like this to put references on a page
% by themselves when using endfloat and the captionsoff option.
\ifCLASSOPTIONcaptionsoff
  \newpage
\fi



% trigger a \newpage just before the given reference
% number - used to balance the columns on the last page
% adjust value as needed - may need to be readjusted if
% the document is modified later
%\IEEEtriggeratref{8}
% The "triggered" command can be changed if desired:
%\IEEEtriggercmd{\enlargethispage{-5in}}

% references section

% can use a bibliography generated by BibTeX as a .bbl file
% BibTeX documentation can be easily obtained at:
% http://mirror.ctan.org/biblio/bibtex/contrib/doc/
% The IEEEtran BibTeX style support page is at:
% http://www.michaelshell.org/tex/ieeetran/bibtex/
\bibliographystyle{IEEEtran}
% argument is your BibTeX string definitions and bibliography database(s)
\bibliography{orchload}
%
% <OR> manually copy in the resultant .bbl file
% set second argument of \begin to the number of references
% (used to reserve space for the reference number labels box)
%\begin{thebibliography}{1}
%\bibitem{IEEEhowto:kopka}
%H.~Kopka and P.~W. Daly, \emph{A Guide to \LaTeX}, 3rd~ed.\hskip 1em plus
%  0.5em minus 0.4em\relax Harlow, England: Addison-Wesley, 1999.
%
%\end{thebibliography}

% biography section
% 
% If you have an EPS/PDF photo (graphicx package needed) extra braces are
% needed around the contents of the optional argument to biography to prevent
% the LaTeX parser from getting confused when it sees the complicated
% \graphics command within an optional argument. (You could create
% your own custom macro containing the \includegraphics command to make things
% simpler here.)
%\begin{IEEEbiography}[{\includegraphics[width=1in,height=1.25in,clip,keepaspectratio]{mshell}}]{Michael Shell}
% or if you just want to reserve a space for a photo:

%\begin{IEEEbiography}[{\includegraphics[width=1in,height=1.25in,clip,keepaspectratio]{img/LPPhoto.jpg}}]{Luis P. Prieto} received his Ph.D. in Information and Communication Technologies from the University of Valladolid. He is a Marie Curie Fellow at the CHILI Lab in the \'Ecole Polytechnique F\'ed\'erale de Lausanne (EPFL). His research interests include classroom orchestration, distributed and augmented paper learning technologies, and the use of wearable technologies to capture and understand complex practice with ICT and use it for reflection. He has authored more than fifty academic publications on these topics, and he is a member of IEEE and ACM.
%\end{IEEEbiography}
%
%
%\begin{IEEEbiography}[{\includegraphics[width=1in,height=1.25in,clip,keepaspectratio]{img/Sharma.png}}]{Kshitij Sharma}
% is ...
%\end{IEEEbiography}
%
%\begin{IEEEbiography}[{\includegraphics[width=1in,height=1.25in,clip,keepaspectratio]{img/Kidzinski.jpg}}]{{\L}ukasz Kidzinski}
% is ...
%\end{IEEEbiography}
%
%\begin{IEEEbiography}[{\includegraphics[width=1in,height=1.25in,clip,keepaspectratio]{img/Dillenbourg.jpg}}]{Pierre Dillenbourg}
% is ...
%\end{IEEEbiography}


\begin{IEEEbiography}[{\includegraphics[width=1in,height=1.25in,clip,keepaspectratio]{img/LPPhoto.jpg}}]{Luis P. Prieto} received his Ph.D. in Information and Communication Technologies from the University of Valladolid (Spain). He is a Marie Curie Fellow at the CHILI Lab in the \'Ecole Polytechnique F\'ed\'erale de Lausanne (EPFL). His research interests include classroom orchestration, distributed and augmented paper learning technologies, and the use of wearable technologies to capture and understand complex practice with ICT and the use of such data for reflection. He has authored more than fifty academic publications on these topics, and he is a member of IEEE and ACM.
\end{IEEEbiography}


\begin{IEEEbiography}[{\includegraphics[width=1in,height=1.25in,clip,keepaspectratio]{img/Sharma.png}}]{Kshitij Sharma}
 is ...
\end{IEEEbiography}

\begin{IEEEbiography}[{\includegraphics[width=1in,height=1.25in,clip,keepaspectratio]{img/Kidzinski.jpg}}]{{\L}ukasz~Kidzinski}
 is ...
\end{IEEEbiography}


\begin{IEEEbiography}[{\includegraphics[width=1in,height=1.25in,clip,keepaspectratio]{img/Dillenbourg.jpg}}]{Pierre Dillenbourg}
 is ...
\end{IEEEbiography}

% You can push biographies down or up by placing
% a \vfill before or after them. The appropriate
% use of \vfill depends on what kind of text is
% on the last page and whether or not the columns
% are being equalized.

%\vfill

% Can be used to pull up biographies so that the bottom of the last one
% is flush with the other column.
%\enlargethispage{-5in}



% that's all folks
\end{document}


